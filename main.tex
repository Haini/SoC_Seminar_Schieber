\documentclass[10pt, journal]{IEEEtran}

\usepackage{url}
\usepackage{cite}
\usepackage{amsmath,amssymb,amsfonts}
\usepackage{algorithmic}
\usepackage{graphicx}
\usepackage{textcomp}
\usepackage{xcolor}

\usepackage[T1]{fontenc} % optional
\usepackage[cmintegrals]{newtxmath}
\usepackage{bm} % optional

\usepackage{glossaries}
\makeglossaries
\loadglsentries{glsEntries}

\title{Dynamic Partial Self-Reconfiguration of Self-Aware Systems}
\author{Constantin Schieber, 01228774}
\begin{document}
\maketitle

\begin{abstract}
    This work focuses on fault aspects of Self-Aware Systems.
    This includes a review of the types of faults that are occurring, methods for their detection and how one mitigate certain faults. 
    One could argue that \glspl{NoC} are Self-Aware too, as the nodes / controllers monitor the behavior of the throughput and can react accordingly to arising situations.
\end{abstract}

\section{Types of Faults}
\subsection{Transient Faults}
\begin{itemize}
    \item \glspl{SEU}, e.g. radiation induced \cite{alkady_fault-tolerant_2014}, \cite{lee_fault-tolerant_2017}
    \begin{itemize}
    \item Change of logic state in memory cell
    \item Commonly tackled by redundancy
    \item Built-in fault detection unit possible
    \end{itemize}
    \item Single Bit Errors (SBEs)
    \item Single Event Transients (SETs)
    \item Address Decoding Faults
\end{itemize}

Faults occur either in the interconnect of the \gls{FPGA} (which uses up to 80\% of the available silicon) or in its actual logic blocks \cite{alkady_fault-tolerant_2014}, \cite{jing_huang_routability_2004}.
\subsection{Permanent Faults}
\begin{itemize}
    \item Time Dependant Dielectric Breakdowns (TDDBs)
    \item Electro Migration
    \item Hot Carrier Effect
\end{itemize}

\section{Detection of Faults}
\subsection{Network on Chip}
\cite{sterpone_new_2012} introduces ways to test the fault tolerance of \gls{NoC}.
\subsection{Monitoring AXI-Core Traffic}
\cite{navas_towards_2015} introduces the basis of cognitive reconfigurable hardware and presents a design that maintains a desired system performance by using \gls{RTR} and self-awareness.
Self-awareness is achieved by monitoring and evaluating critical AXI-core metrics.

\subsection{Duplication with Compare}
\cite{alkady_dynamic_2015} proposes a fault detection technique to detect open faults, Stuck-At faults and \glspl{SEU}.

\section{Mitigation of Faults}
\subsection{Network on Chip - make fault-tolerant per design}
\cite{yesil_fpga_2016} adds additional network resources to a non-fault-tolerant design to mitigate interconnect faults.

\cite{lu_fault-tolerant_2015} provides a case study on implementing a fault-tolerant routing algorithm and its monitoring mechanism.
\subsection{Redundancy by need based replication}
\cite{glein_self-adaptive_2014} tackles the problem of radiation by the creation of redundant modules - based on currently measured \glspl{SEU} rates in on-chip memories.
\subsection{Dynamic Partial Self-Reconfiguration}
\cite{alkady_dynamic_2015} goes on to propose system recovery by re-instantiating defective modules into the Partially Reconfigurable block.

\cite{shanker_enhancing_nodate} proposes a similar solution in the context of \glspl{ECU}.

\cite{sharma_run-time_2018} shows an approach on how to decide which configuration is currently desired in a multi-modal / multi-task \gls{SoPC}.
\printglossaries 

\bibliographystyle{IEEEtran}
\bibliography{IEEEabrv,SocSeminar}

\end{document}