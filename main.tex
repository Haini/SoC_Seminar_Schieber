\documentclass[10pt, journal]{IEEEtran}

\usepackage{url}
\usepackage{cite}
\usepackage{amsmath,amssymb,amsfonts}
\usepackage{algorithmic}
\usepackage{graphicx}
\usepackage{textcomp}
\usepackage{xcolor}

\usepackage[T1]{fontenc} % optional
\usepackage[cmintegrals]{newtxmath}
\usepackage{bm} % optional

\usepackage{glossaries}
\makeglossaries
\loadglsentries{glsEntries}

\title{Dynamic Partial Self-Reconfiguration of Self-Aware Systems}
\author{Constantin Schieber, 01228774}
\begin{document}
\maketitle

\begin{abstract}
    This work focuses on fault aspects of Self-Aware Systems.
    This includes a review of the types of faults that are occurring, methods for their detection and how one mitigate certain faults. 
    One could argue that \glspl{NoC} are Self-Aware too, as the nodes / controllers monitor the behaviour of the throughput and can react accordingly to arising situations.
\end{abstract}

\section{Introduction}
\gls{DPR} for fault mitigation can be split into two major fields of appliance. 
The first field discusses the mitigation of so called \textit{internal} faults.
This encompasses faults that interfere with the current functionality of the \gls{FPGA} itself, e.g. permanent faults and transient faults.
The second area focuses on the mitigation of faults that are present in external devices by the provision of redundancy.

The rest of this work is organized as follows.
Section \ref{InternalFaults} introduces different causes of faults within the \gls{FPGA} and shows strategies to maintain the desired behaviour.
Section \ref{ExternalFaults} describes scenarios where external faults may happen and how they can be mitigated with the usage of minimal hardware resources. 

\section{\gls{DPR} for Internal Fault Mitigation}\label{InternalFaults}
\glspl{FPGA} provide high flexibility and good performance in many areas that require a high amount of concurrent computing and therefore benefit from a dedicated hardware implementation.
These properties prove useful for applications in the domain of aerospace and help to reduce the cost for satellites and spacecraft while providing a high flexibility. 
But \glspl{FPGA} are especially vulnerable to cosmic radiation which can create a multitude of internal errors in the \gls{FPGA} and thereby breaking its intended functionality \cite{ito_total_2015}.
While there are radiation hardend \glspl{FPGA} available on the market, the mass of a space embedded system is still composed of 80\% radiation shielding.
This shielding is still not able to provide a perfect protection from radiation.
Therefore, solutions for fault mitigation in aerospace \glspl{FPGA} need to be developed on the architectural level instead of the physical level. 
This may also provide the advantage of lower requirements for radiation shielding and therefore a reduced cost for payload on rocket launches.

\subsection{Types of Internal Faults}
There are different types of faults that may occour in the lifetime of an FPGA, a short overview over the most relevant ones is given in the following.
\par
\textbf{Transient Faults}
\begin{itemize}
    \item \glspl{SEU}, e.g. radiation induced \cite{alkady_fault-tolerant_2014}, \cite{lee_fault-tolerant_2017}
    \begin{itemize}
    \item Change of logic state in memory cell
    \item Commonly tackled by redundancy
    \item Built-in fault detection unit possible
    \end{itemize}
    \item Single Bit Errors (SBEs)
    \item Single Event Transients (SETs)
    \item Address Decoding Faults
\end{itemize}

Faults occur either in the interconnect of the \gls{FPGA} (which uses up to 80\% of the available silicon) or in its actual logic blocks \cite{alkady_fault-tolerant_2014}, \cite{jing_huang_routability_2004}.
\par
\textbf{Permanent Faults}
\begin{itemize}
    \item Time Dependant Dielectric Breakdowns (TDDBs)
    \item Electro Migration
    \item Hot Carrier Effect
\end{itemize}
\subsection{Mitigation Strategies for Internal Faults}

\section{\gls{DPR} for External Fault Mitigation}\label{ExternalFaults}
\subsection{Types of External Faults}
\subsection{Mitigation Strategies for Externall Faults}

\section{Mitigation of Faults}
\subsection{Network on Chip - make fault-tolerant per design}
\cite{yesil_fpga_2016} adds additional network resources to a non-fault-tolerant design to mitigate interconnect faults.

\cite{lu_fault-tolerant_2015} provides a case study on implementing a fault-tolerant routing algorithm and its monitoring mechanism.
\subsection{Redundancy by need based replication}
\cite{glein_self-adaptive_2014} tackles the problem of radiation by the creation of redundant modules - based on currently measured \glspl{SEU} rates in on-chip memories.
\subsection{Dynamic Partial Self-Reconfiguration}
\cite{alkady_dynamic_2015} goes on to propose system recovery by re-instantiating defective modules into the Partially Reconfigurable block.

\cite{shanker_enhancing_nodate} proposes a similar solution in the context of \glspl{ECU}.

\cite{sharma_run-time_2018} shows an approach on how to decide which configuration is currently desired in a multi-modal / multi-task \gls{SoPC}.
\printglossaries 

\bibliographystyle{IEEEtran}
\bibliography{IEEEabrv,SocSeminar}

\end{document}
