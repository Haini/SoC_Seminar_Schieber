%\tableofcontents

\section{Introduction}\label{Introduction}
\glspl{FPGA} provide us with flexible options to reconfigure functionality at run-time. 
This flexibility can be exploited to increase fault resistance which in turn reduces the need for hardware redundancy. 

\gls{DPR} for fault mitigation can be split into two major fields of appliances. 
The first field discusses the mitigation of faults that occur within the \gls{FPGA}.
These faults interfere with the intended functionality of the tasks executed on the \gls{FPGA} itself, e.g. permanent and transient faults. 
The second area focuses on the mitigation of faults that are present in external devices by the provision of redundancy in the form of the \gls{FPGA}.
In the following, this work will refer to these two types of faults as \textit{internal} and \textit{external} faults, respectively.

The rest of this work is organized as follows.
Section \ref{InternalFaults} introduces different types of internal faults and a generally applicable fault mitigation flow in the context of \gls{DPR}.
After this, different architectures and strategies for fault mitigation with \gls{DPR} are introduced in Section \ref{InternalFaultsArch}.
This involves architectures that focus solely on the aspect of moving functionality from faulty to healthy tiles as well as solutions like \glspl{NoC}, that come with built in fault tolerance. 
For \glspl{NoC}, first we cover a security relevant scenario where faults are introduced with malicious intent by a third party through partial reconfiguration of a modified \glspl{IP}.
Second, we show how \glspl{NoC} can deal with internal faults that affect routing within the network.  
Section \ref{ExternalFaults} describes scenarios where external faults may happen and how they can be mitigated with the usage of minimal hardware resources. 
Section \ref{Conclusion} gives concluding remarks on the surveyed applications and an estimate on future developments.

\begin{table*}
    \caption{Literature on the usage of \gls{DPR} for fault-tolerance.}
    \begin{tabular*}{\textwidth}{@{\extracolsep{\fill}}llccc}
        \toprule
       \textbf{Authors} & \textbf{Mitigation Method} & \textbf{Fault Detection Method} & \textbf{Novelty} & \textbf{Overhead} \\
       \midrule
       \cite{bolchini2007} Bolchini et al. 2007        & Spare Tile Architecture      & TMR & \\
       \cite{lameres_radsat_2015} LaMeres et al. 2015       & Spare Tile Architecture      & TMR & \\
       \cite{wilson_hybrid_2017} Wilson et al. 2017         & Spare Tile Architecture      & TMR & \\
       \cite{martins_dynamic_2018} Martins et al. 2018      & Spare Tile Architecture      & TMR & \\
       \cite{martins_tmr_2015} Martins et al. 2015          & Spare Tile Architecture      & TMR & \\
       \cite{wang_dynamic_2018} Wang et al. 2018            & Task Based Architecture      & Scrubbing & \\
       \cite{sharma_run-time_2018} Sharma et al. 2018       & Task Based Architecture      & - & \\
       \cite{sharma_run-time_2018} Sharma et al. 2018       & Task Based Architecture      & - & \\
       \cite{kadri_survey_2019} Kadri et al. 2019           & NoC                          & &\\
       \cite{majer_packet_2005} Majer et al. 2005           & NoC                          & &\\
       \cite{wehbe_secure_2016} Wehbe et al. 2016           & NoC                          & &\\
       \cite{crdl_fail-safe_2002} Crdl et al. 2002          & External Redundancy           & &\\
       \cite{kastil2012} Kastil et al. 2012                 & Spare Tile Architecture      & TMR, DWC, CED  &\\
       \cite{zhang2013} Zhang et al. 2013                   &            & &\\
       \cite{davis2014} Davis et al. 2014                   &            & &\\
       \cite{dicarlo2014} Dicarlo et al. 2014               &            & &\\
       \cite{kourfali2019} Kourfali et al. 2019             &            & &\\
       \cite{frenkel2015} Frenkel et al. 2015             &            & Temporal Redundancy &\\
       \cite{alkady_integration_2016} Alkady et al. 2016             & Temporal Redundancy             & &\\
       \bottomrule
    \end{tabular*}
    Text
\end{table*}
