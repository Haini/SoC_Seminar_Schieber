%\tableofcontents

\section{Introduction}\label{Introduction}
\glspl{FPGA} provide us with flexible options to reconfigure functionality at run-time. 
This flexibility can be exploited to increase fault resistance which in turn reduces the need for hardware redundancy. 

\gls{DPR} for fault mitigation can be split into two major fields of appliances. 
The first field discusses the mitigation of faults that occur within the \gls{FPGA}.
These faults interfere with the intended functionality of the tasks executed on the \gls{FPGA} itself, e.g. permanent and transient faults. 
The second area focuses on the mitigation of faults that are present in external devices by the provision of redundancy in the form of the \gls{FPGA}.
In the following, this work will refer to these two types of faults as \textit{internal} and \textit{external} faults, respectively.

The rest of this work is organized as follows.
A general overview over the existing literature and its main methodologies is given in \ref{sec:literatureOverview}.
Section \ref{InternalFaults} introduces different types of internal faults and a generally applicable fault mitigation flow in the context of \gls{DPR}.
After this, different architectures and strategies for fault mitigation with \gls{DPR} are introduced in Section \ref{InternalFaultsArch}.
This involves architectures that focus solely on the aspect of moving functionality from faulty to healthy tiles as well as solutions like \glspl{NoC}, that come with built in fault tolerance. 
For \glspl{NoC}, first we cover a security relevant scenario where faults are introduced with malicious intent by a third party through partial reconfiguration of a modified \glspl{IP}.
Second, we show how \glspl{NoC} can deal with internal faults that affect routing within the network.  
Section \ref{ExternalFaults} describes scenarios where external faults may happen and how they can be mitigated with the usage of minimal hardware resources. 
Section \ref{Conclusion} gives concluding remarks on the surveyed applications and an estimate on future developments.

\begin{table*}
    \caption{Literature on the usage of \gls{DPR} for fault-tolerance.}
    \label{tbl:literatureOverview}
    \begin{tabular*}{\textwidth}{@{\extracolsep{\fill}}lll}
        \toprule
       \textbf{Authors} & \textbf{Fault Mitigation Method} & \textbf{Fault Detection Method} \\
       \midrule
       \cite{bolchini2007} Bolchini et al. 2007             & Spare Tile Architecture      & TMR  \\
       \cite{kastil2012} Kastil et al. 2012                 & Spare Tile Architecture      & TMR, DWC, CED  \\
       \cite{zhang2013} Zhang et al. 2013                   & Spare Tile Architecture      & Not specified \\
       \cite{dicarlo2014} Dicarlo et al. 2014               & Spare Tile Architecture      & Error Signals, Tests \\
       \cite{davis2014} Davis et al. 2014                   & Spare Tile Architecture      & Algorithm Based Fault Tolerance \\
       \cite{lameres_radsat_2015} LaMeres et al. 2015       & Spare Tile Architecture      & TMR  \\
       \cite{martins_tmr_2015} Martins et al. 2015          & Spare Tile Architecture      & TMR  \\
       \cite{wilson_hybrid_2017} Wilson et al. 2017         & Spare Tile Architecture      & TMR  \\
       \cite{martins_dynamic_2018} Martins et al. 2018      & Spare Tile Architecture      & TMR  \\
       \cite{wang_dynamic_2018} Wang et al. 2018            & Task Based Architecture      & Scrubbing \\
       \cite{sharma_run-time_2018} Sharma et al. 2018       & Task Based Architecture      & - \\
       \cite{kadri_survey_2019} Kadri et al. 2019           & NoC - Routing, Remapping, Redundancy                        & - \\
       \cite{wehbe_secure_2016} Wehbe et al. 2016           & NoC - Routing               & Timeout\\
       \cite{frenkel2015} Frenkel et al. 2015               & Localized Scrubbing       & Forward Temporal Redundancy \\
       \cite{nunes_improving_2016} Nunes et al. 2016        & Localized Scrubbing         & - \\
       \cite{kourfali2019} Kourfali et al. 2019             & Localized Scrubbing           & TMR \\
       \cite{zhao_fine-grained_2018} Zhao et al. 2018       & Fine grained reconfiguration & TMR \\
       \cite{alkady_integration_2016} Alkady et al. 2016    & \gls{DPR}             & TMR, Temporal Redundancy \\
       \cite{crdl_fail-safe_2002} Crdl et al. 2002          & External Redundancy           & Bus Monitoring, Watchdog \\
       \bottomrule
    \end{tabular*}
\end{table*}

\section{Literature Overview}\label{sec:literatureOverview}
Table \ref{tbl:literatureOverview} gives an overview over the existing literature and tries to classify the existing work by the used \textit{Fault Mitigation Method} and the \textit{Fault Detection Method}.
One can see that the prevalent strategy for Fault Mitigation is the \textit{Spare Tile Architecture}. 
This makes sense, as it is straight forward to implement with most of the standard vendor tool-chains.
An architecture was classified as \textit{Spare Tile Architecture} if it was reserving extra space for potentially occurring faults, consequently moving a faulty module to the healthy extra space.
Designs with this architecture mostly utilize \gls{TMR} as a \textit{Fault Detection Method} as it is easy and reliable to move a faulty module to the reserved space without compromising functionality.
Furthermore, these solutions can handle transient and permanent faults equally well.
The disadvantage of these methods in general is that they introduce an overhead in terms of used resources and storage for partial bitstreams.
Solutions to this are introduced in Section \ref{sec:SpareTileArchitecture}. 
They are also inflexible or at least not easy to generalize.

A solution to this is introduced with the \textit{Task Based Architecture}.
These architectures focus on a \gls{RTOS} based approach, where resilient modules are modelled as tasks that may be placed anywhere, anytime within the \gls{RP}. 
The listed works don't focus on fault detection, as it should be managed on the level of the \gls{RTOS}.
With a standardized communication interface and an OS-like scheduler, faults can be mitigated by re-scheduling modules to healthy tiles as needed. 
This makes the approach much more general, albeit it may compromise hard real-time requirements.

\glspl{NoC} already heave the inherent property of a standardized communication interface which may also be utilized to increase fault tolerance.
Re-routing of data-paths to avoid faulty or malicious nodes proves to be a useful means for handling transient and permanent faults.

Localized scrubbing is used to reduce scrubbing times, but requires the development of non-vendor tools and the modification of partial reconfiguration controllers.
It profits from a finer grained localization of faults.
While this approach reduces resource usage, it is not able to handle permanent faults. 