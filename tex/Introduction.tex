%\tableofcontents

\section{Introduction}\label{Introduction}
\glspl{FPGA} provide us with flexible options to reconfigure functionality at run-time. 
This flexibility can be exploited to increase fault resistance which in turn reduces the need for hardware redundancy. 

\gls{DPR} for fault mitigation can be split into two major fields of appliances. 
The first field discusses the mitigation of faults that occur within the \gls{FPGA}.
These faults interfere with the intended functionality of the tasks executed on the \gls{FPGA} itself, e.g. permanent and transient faults. 
The second area focuses on the mitigation of faults that are present in external devices by the provision of redundancy in the form of the \gls{FPGA}.
In the following, this work will refer to these two types of faults as \textit{internal} and \textit{external} faults, respectively.

The rest of this work is organized as follows.
Section \ref{InternalFaults} introduces different causes of faults within the \gls{FPGA} and shows strategies to maintain the desired behaviour.
Actions to increase security and resilience \glspl{NoC} are presented in Section \ref{sec:NoC}.
Section \ref{ExternalFaults} describes scenarios where external faults may happen and how they can be mitigated with the usage of minimal hardware resources. 
Section \ref{Conclusion} gives concluding remarks on the surveyed applications and an estimate on future developments.
